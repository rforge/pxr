\documentclass[10pt,twoside,spanish]{article}
%%%%%%%%%%%%%%%%%%%%%%%%%%%%%%%%%%%%%%%%%%%%%%%%%%%%%%%%%%%%%%%%%%%%%%%%%%%%%%%%%%%%%%%%%%%%%%%%%%%%%%%%%%%%%%%%%%%%%%%%%%%%%%%%%%%%%%%%%%%%%%%%%%%%%%%%%%%%%%%%%%%%%%%%%%%%%%%%%%%%%%%%%%%%%%%%%%%%%%%%%%%%%%%%%%%%%%%%%%%%%%%%%%%%%%%%%%%%%%%%%%%%%%%%%%%%
\usepackage[ansinew]{inputenc}
\usepackage[spanish,english]{babel}
\usepackage[T1]{fontenc}
\usepackage{epsfig}
\usepackage{amsmath}
\usepackage{array}
\usepackage{amssymb}
\usepackage{epsf}
\usepackage{graphicx}
\usepackage{latexsym,bm}
\usepackage{amsthm}
\usepackage{amsfonts}
\usepackage{epsfig}
\usepackage{longtable}
\usepackage{multicol}
\usepackage{fancyhdr}
\usepackage{caption}
\usepackage[dvips,dvipsnames,usenames]{color}
\usepackage[dvips,dvipsnames,usenames]{colortbl}
\usepackage{setspace}

%OPL: añadido
\usepackage{hyperref}
\hypersetup{
    bookmarks=true,         % show bookmarks bar?
    unicode=true,          % non-Latin characters in Acrobat’s bookmarks
    bookmarksnumbered=false,
    bookmarksopen=false,
    breaklinks=true,
    backref=true,
    pdftoolbar=true,        % show Acrobat’s toolbar?
    pdfmenubar=true,        % show Acrobat’s menu?
    pdffitwindow=false,     % window fit to page when opened
    pdfstartview={FitH},    % fits the width of the page to the window
    %pdftitle={}
    %pdfauthor={},     % author
    %pdfsubject={},   % subject of the document
    pdfcreator={AucTeX/Emacs},   % creator of the document
    pdfproducer={LaTeX}, % producer of the document
    %pdfkeywords={}, % list of keywords
    pdfnewwindow=true,      % links in new window
    pdfborder={0 0 0},
    colorlinks=true,       % false: boxed links; true: colored links
    linkcolor=BrickRed,          % color of internal links
    citecolor=BrickRed,        % color of links to bibliography
    filecolor=black,      % color of file links
    urlcolor=Blue           % color of external links 
}

\setcounter{MaxMatrixCols}{10}
%TCIDATA{OutputFilter=LATEX.DLL}
%TCIDATA{Version=5.00.0.2552}
%TCIDATA{<META NAME="SaveForMode" CONTENT="1">}
%TCIDATA{LastRevised=Wednesday, January 21, 2009 20:03:48}
%TCIDATA{<META NAME="GraphicsSave" CONTENT="32">}
%TCIDATA{CSTFile=article.cst}

\newtheorem{theorem}{Teorema}[section]
\newtheorem{proposition}{Proposición}[section]
\newtheorem{lemma}{Lema}[section]
\newtheorem{corollary}{Corolario}[section]
\newtheorem{definition}{Definición}[section]
\newtheorem{remark}{Observación}[section]
\newtheorem{example}{Ejemplo}[section]
\newtheorem{conclusion}{Conclusión}[section]
\newtheorem{conjecture}{Conjetura}[section]
\newtheorem{notation}{Notación}[section]
\newtheorem{exercise}{Ejercicio}[section]
\newtheorem{problem}{Problema}[section]
\renewenvironment*{proof}[1][Demostración]{\noindent \textbf{#1.} }{\ \rule{0.5em}{0.5em}}
\numberwithin{equation}{section}
\renewcommand{\theequation}{\thesection.\arabic{equation}}
\renewcommand{\thefigure}{\arabic{figure}}
\font\gotica=eufm9 scaled\magstep2
\font\gotsmall=eufm8
\renewcommand{\baselinestretch}{1.1}
\footnotesep 0.2cm
\skip\footins 0.3cm
\abovecaptionskip 0cm
\belowcaptionskip -0.2cm
\renewcommand*{\refname}{Referencias}
\hyphenation{con-ve-nien-te te-o-re-mas}
%%\input{tcilatex} ¿¿¿??? OPL: comentado

\begin{document}

\title{%
%TCIMACRO{\TeXButton{TeX field}{\vspace*{-2cm}}}%
%BeginExpansion
\vspace*{-2cm}%
%EndExpansion
Title of the Article in English\thanks{%
Escribir a pie de página un Título Corto si el Título (Largo) supera los 100
caracteres contando los espacios en blanco.}%
%TCIMACRO{\TeXButton{TeX field}{\vspace*{-0.2cm}}}%
%BeginExpansion
\vspace*{-0.2cm}%
%EndExpansion
}
\author{Nombre y Apellidos Autor 1, Nombre y Apellidos Autor 2\thanks{%
Corresponding Author.} \\
%EndAName
Centro (Departamento, Instituto de Investigación, etc.) \\
Institución (Universidad, Empresa, etc.)\\
E-mail Autor 1, E-mail Autor 2 \and Nombre y Apellidos Autor 3 \\
%EndAName
Centro (Departamento, Instituto de Investigación, etc.) \\
Institución (Universidad, Empresa, etc.)\\
E-mail Autor 3}
\date{%
%TCIMACRO{\TeXButton{Espacio negativo}{\vspace*{-1cm}}}%
%BeginExpansion
\vspace*{-1cm}%
%EndExpansion
}
\maketitle

\begin{abstract}
This is the abstract of the article. It should be no longer than ten lines,
and it must be written in English.\medskip

\noindent \textbf{Keywords:} Keyword 1, Keyword 2, ...

\noindent \textbf{AMS Subject Classifications}: 90CXX, 90CXY, ... (see
http://www.ams.org/msc/)
\end{abstract}

%TCIMACRO{\TeXButton{Artículo en español}{\selectlanguage{spanish}}}%
%BeginExpansion
\selectlanguage{spanish}%
%EndExpansion

\section{Título\textbf{\ Sección}}

Esto es una muestra del formato que debe ser utilizado para un artículo de
la revista BEIO\footnote{%
Los espacios entre párrafos y los cortes de palabras se ajustarán
posteriormente en la edición de la revista.}.

\begin{remark}
Los artículos para la sección \emph{Opiniones sobre la Profesión} no
necesitan Abstract, Keywords ni AMS Subject classifications. Los autores de
esta sección deben mandar una foto personal en formato PostScript.\FRAME{%
dtbpFU}{1.2185in}{1.4806in}{0pt}{\Qcb{{}}}{}{leibnitz.eps}{\special{language
"Scientific Word";type "GRAPHIC";maintain-aspect-ratio TRUE;display
"USEDEF";valid_file "F";width 1.2185in;height 1.4806in;depth
0pt;original-width 2.6489in;original-height 3.2232in;cropleft "0";croptop
"1";cropright "1";cropbottom "0";filename 'leibnitz.eps';file-properties
"XNPEU";}}
\end{remark}

El artículo puede contener todo tipo de símbolos matemáticos compatibles con
Tex. Los tamaños de las figuras deben estar ajustados adecuadamente y sus
ficheros deberán enviarse junto con el fichero Tex principal.

\FRAME{dtbpFU}{0.4912in}{0.4921in}{0pt}{\Qcb{Logotipo de la SEIO}}{}{Figure}{%
\special{language "Scientific Word";type "GRAPHIC";maintain-aspect-ratio
TRUE;display "USEDEF";valid_file "T";width 0.4912in;height 0.4921in;depth
0pt;original-width 8.9802in;original-height 8.9906in;cropleft "0";croptop
"1";cropright "1";cropbottom "0";tempfilename
'KC88UV00.bmp';tempfile-properties "XPR";}}

A continuación podemos ver el estilo de los teoremas, las proposiciones, etc.

\begin{theorem}
Enunciado del teorema:%
\begin{equation}
E=mc^{2}  \label{Equation 1}
\end{equation}
\end{theorem}

\begin{proof}
La demostración va aquí.
\end{proof}

\begin{proposition}
Dos ecuaciones en diferentes líneas, la primera sin numerar:%
\begin{gather}
\Delta u=0  \notag \\
x^{n}+y^{n}=z^{n}  \label{Equation 2b}
\end{gather}
\end{proposition}

\subsection{Título Subsección}

\begin{theorem}
Otro teorema.
\end{theorem}

Algunos ítems:

\begin{itemize}
\item Primer ítem

\item Segundo ítem

\begin{itemize}
\item Ítem 1 del segundo nivel

\item Ítem 2 del segundo nivel
\end{itemize}
\end{itemize}

Algunos ítems numerados:

\begin{enumerate}
\item Enumeración 1

\begin{enumerate}
\item Enumeración 1 del segundo nivel
\end{enumerate}
\end{enumerate}

\section{Título Sección\label{Sec 2}}

\subsection{Título Subsección}

\begin{lemma}
Enunciado del lema:%
\begin{equation}
a\leq b~\wedge ~b\leq c\Rightarrow a\leq c  \label{Propiedad transitiva}
\end{equation}
\end{lemma}

\section{Acerca de las referencias}

Las referencias en el texto se deben citar por los apellidos de los autores
seguidos del año de publicación entre paréntesis. Algunos ejemplos:

En Apellido Autor 1 (2000) se estudia ...

El concepto está definido en Apellido Autor 2 y Apellido Autor 3 (2001).

Este tema ha sido estudiado ampliamente en Apellido Autor 1 (2000), Apellido
Autor 2 y Apellido Autor 3 (2001), Apellido Autor 4 et al. (2002).

La lista de referencias sólo debe incluir trabajos que están citados en el
texto y que han sido publicados o aceptados para su publicación. Las
comunicaciones personales y los trabajos no publicados sólo se deben
mencionar en el texto. No utilizar notas al pie para sustituir una lista de
referencias.

La lista de referencias debe ser ordenada alfabéticamente por los apellidos
del primer autor de cada trabajo.

Utilizar siempre la abreviatura del nombre de la revista de acuerdo al ISSN
List of Title Word Abbreviations, véase www.issn.org/2-22661-LTWA-online.php

\begin{thebibliography}{9}
\bibitem{Etiqueta 1} Apellido seguido de inicial del nombre. (Año). Nombre
del artículo. \emph{Nombre de la revista abreviado}, \textbf{Número del
Volumen}, Numero de la página inicial-Número de la página final.

\bibitem{Etiqueta 2} Apellido2 N2., y Apellido3 N3. (Año). \emph{Nombre del
Libro con Iniciales en Mayúsculas}, Editorial, Ciudad de Publicación (País).

\bibitem{Etiqueta 3} Apellido4 N4., Apellido5 N5., y Apellido6 N6. (Año). Cap%
ítulo de libro. En: \emph{Nombre del Libro con Iniciales en Mayúsculas},
Editorial, Ciudad de Publicación (País).

\bibitem{Etiqueta 4} Apellido7 N7., Apellido8 N8., Apellido9 N9., et al. (Añ%
o). Nombre del artículo publicado por DOI. \emph{Nombre de la revista
abreviado}, Doi: Número de Doi.

\bibitem{Etiqueta 5} Apellido seguido de inicial del nombre. (Año). Nombre
del documento online. En: www.WebPage.com
\end{thebibliography}

%TCIMACRO{%
%\TeXButton{Acerca de los autores}{\subsection*{Acerca de los autores}}}%
%BeginExpansion
\subsection*{Acerca de los autores}%
%EndExpansion

Breve CV (máximo 10 líneas) por autor.

\textbf{Autor 1} es profesor en ... y sus líneas de investigación son ...

\end{document}
